\section{Conclusions}\label{conclusions}
The designed NLP-based methodology presented in this paper digitizes the RA by accurately extracting its contents and classifying its sub-articles as \textit{standard clauses}, \textit{variations} and \textit{customized texts}. The test phase allows to evaluate the results obtained from two types of experiments conducted to measure the accuracy to establish that on the one hand, the accuracy determination based on a simple inspection at sub-article level determines for the first sample roaming agreement, and accuracy of 80.9\% in the classification of \textit{standard clauses}, an accuracy of 82.9\% in the classification of \textit{variations} and accuracy of 80\% in the classification of \textit{customized texts}. These results are improved for the second sample RA with an accuracy of 92.9\% in the classification of \textit{standard clauses}, an accuracy of 87.5\% in the classification of \textit{variations} and an accuracy of 91.2\% in the classification of \textit{custom texts}. On the other hand, the determination of the accuracy based on the comparison of symbols determines for the first roaming agreement sample, an accuracy of 84.7\% of the total of 72 analyzed sub-articles with a common percentage of favorable symbols. This result is also improved for the second roaming agreement sample with an accuracy of 91.9\% of the total of 86 sub-articles analyzed with a common percentage of favorable symbols. Therefore, the results demonstrate the feasibility of applying the proposed methodology. As part of a process of continuous improvement of the designed methodology within future research lines we aim to conduct other accuracy tests applying other text similarity types, such as cosine similarity, to improve the results obtained.

The proposed NLP Engine constitutes a part of a project that has as the main objective of transforming the current Telecommunication Roaming Agreement drafting and negotiation process into a digitalized version based on the transparency promoted by blockchain technology, future research works include the design, development, and evaluation of the rest of the sub-systems that mainly include the smart contracts that automate the negotiation process, as well as the integration with the developed NLP Engine.