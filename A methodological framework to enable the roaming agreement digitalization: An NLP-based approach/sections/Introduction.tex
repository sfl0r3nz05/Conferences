\section{Introduction}
The roaming service maintains the persistent connectivity of subscribers in different networks and locations. Roaming describes the capability of a subscriber to access mobile services offered by the visited public mobile network (VPMN) through the home public mobile network (HPMN), when roaming outside the coverage range of the HPMN \cite{Tanaka2013}. However, before ensuring persistent connectivity in VPMN the Mobile Network Operators (MNOs) must reach an agreement regarding the technical, commercial and legal relationships known as the Roaming Agreement (RA). In this regard, it is possible to establish three stages within the RA: \textit{drafting}, \textit{testing} and \textit{implementation}. The testing and implementation phases are straight forward as they are related with implementing what is agreed on the drafting phase. However, the drafting phase is ambiguous and involves multiple iterations of exchange of draft text and proposed articles between both parties, i.e. the MNOs. 

Therefore, in order to reduce the exerted efforts and standardize the technical, commercial and legal aspects of the RA, the GSM Association broadly outlines the content of such RA in standardized form for its members \cite{Ferwerda2018}. 
In addition, many organizations attempt to unify RAs. For example, Rocco\footnote{Rocco Group: https://www.rocco.group}, which is a company specialized in telecommunication reports including roaming agreements, provides a list of the most commonly used GSMA standards. It summarizes these standards as follows \cite{ROCCO2017}: (1) AA.12 constitutes the permanent reference document; (2) AA.13 contains the common annexes with operational information (e.g., information on tap file, billing data, settlement procedure, customer care, fraud, etc.) and (3) AA.14 involves the individual annexes containing information about the operator (e.g., contact details of the roaming team, fraud team, IREG team, TADIG team, etc.).

While it is true that it is not mandatory to follow the standards proposed by the GSMA organization, according to authoritative voices in the field of negotiating RA drafting, most MNOs follow them strictly \cite{ROCCO2017a}. Therefore, the first point to consider in the RA drafting is how far it is deviated from the GSMA's proposed standards. Thus, during the drafting process of the agreement, the parties should analyze the sub-articles contained in the GSMA standard templates. This necessitates discretizing the bulk of the roaming agreement template and classifying each word/clause from an existing classes as follows:

\begin{enumerate}
\item Specify the value of certain \textit{variables} that are found in a certain text, such as dates, names of MNOs, locations and others.
\item Introduce certain \textit{variations} in the articles/sub-articles, usually identified as part of the text with different paraphrasing.
\item Leave an article/sub-article as found in the template (RA draft); thereby establishing a \textit{standard clause}.
\item Introduce completely new articles/sub-articles that respond to particular interests by constituting \textit{customized texts}.
\end{enumerate}

However, the \textit{drafting} of a RA goes through a complex negotiation process in which, at present, the parties still use asynchronous flows such as e-mail or even regular mail to exchange the information. Thus, this traditional negotiation process has multiple drawbacks including lacks of transparency, which can lead to violations of the RA by MNOs. In addition, this process is laborious and time consuming. As experts point out, the entire process can take up to one month, depending on the responsiveness of the MNOs \cite{ROCCO2017a}. Therefore, it is necessary to provide a transparent digitalization system for RA drafting negotiations, which guarantees transparency and shortening the negotiation time, which guarantees transparency and shortening the negotiation time, which could be in days or even weeks. Hence, this work develops a framework that exploits the advancement in the Natural Language Processing (NLP) discipline and use it as a legal text in the digitalization engine. This NLP Engine constitutes the starting point for the digitalization of the negotiation process towards a transparent drafting of Roaming Agreements. The proposed NLP Engine analyzes articles and sub-articles of the RA by determining the existence of variables, variations, standard clauses and customized texts. To do so, the proposed NLP Engine relies on the advancement of NLP techniques such as Named Entity Recognition (NER) and Part of Speech (POS) in drafting RAs.

% using the unstructured data information discovery tool \textit{Amazon Comprehend} and, on the other hand, by establishing an comprehensive comparison between texts based on \textit{similarity} determination techniques. The procedures for integrating tools, NLP techniques, pre-processing and post-processing constitute a methodological framework designed for the use case and included within an unstructured text processing tool. This article includes not only the design phase, but also the implementation and evaluation of the NLP Engine tool.

% The rest of this manuscript is structured as follows: Section 2 presents the related work in order to determine the novelty of the proposed topic. Section 3 establishes the designed methodology. The implementation of our system are described in Section 4. Section 5 discusses the results of conducted experiments. Finally, the conclusions of the manuscript are included in Section 6.