\section{Related work}\label{sec:related}

The existing work that is related to the RAs can be categorized into two main categories. The first category encompasses the work, which focused on transparent digitalization of RAs~\cite{9369516, 9024541, Huillet2020}, whereas the second category includes the works that apply existing NLP techniques-based text processing systems~\cite{8487847, 9138070, 9104105}.

Both in the scientific literature and business environments, there are important approaches to RA digitalization. Thus, the work discussed in~\cite{9369516} proposes a dynamic RA between the Local 5G Operator and the MNO. The interaction between the two entities takes place through an Ethereum based platform. Despite the contributions of this work in the RA \textit{implementation} phase, the fact of performing the implementation on the Ethereum network implies that the participating MNOs will bear the high cost of gas fees for the transactions. Additionally, although the authors justify the latency given that they repeated the experiment 100 times, the proof of work nature of the network implies drafting actions are likely to suffer congestion anytime. A second approach focuses uniquely on the billing of the services obtained as a result of the RA is reported in~\cite{9024541}. This agreement is incorporated as part of a chaincode of a Hyperledger Fabric Blockchain (HFB) network so the work contributes significantly to the digitalization process of the RA, allowing for a faster, more seamless process in which payments can be requested and obtained quickly due to less need for manual intervention. Although this work also constitutes an important contribution regarding the use of HFB in the field of telecommunications to address problems around Managing RAs, Inter-carrier Settlements, and Mobile Number Portability, its approach is one of review and proposal, however, it lacks an implementation section that would allow demonstrating the feasibility of the proposals. The contextualization of this system in the business environment is proposed by important MNOs such as Telefonica, Deutsche Telekom, and Vodafone which use blockchain for Roaming settlement within the framework of the RA between the parties~\cite{Huillet2020}. This proposal is also of considerable value in the business environment, however, its focus relates only to the \textit{implementation} phase of the RA and not to the \textit{drafting} phase. 

Additionally, the scientific literature addresses text processing systems based on NLP techniques in domains such as the judiciary. Thus, the approach introduced in~\cite{8487847} addresses the process of digitalization in the judicial sectors from archives of judicial records for which the authors have designed a text analysis tool that includes grammatical analysis of documents in English based on NLP techniques. The proposed article also constitutes an important contribution in terms of the design of the linguistic analysis based on NLP techniques. However, it lacks formal evaluation in terms of accuracy of NLP implementation. In addition, the work reported in~\cite{9138070} applies NLP-based processing techniques for information retrieval from spreadsheets and describes technologies for storing and retrieving database information. NLP techniques such as sentence tokenization, word tokenization, removing stopwords and lemmatization are part of the parsing stage of the work. The work constitutes a valuable contribution in terms of design and implementation of the tool, however, it lacks demonstration of the feasibility of use, as well as determination of the accuracy of the NLP techniques applied. Finally, authors of~\cite{9104105} propose a methodology for implementing sentimental analysis using Amazon Comprehend. This system performs an audio-to-text transcription and then performs the processing of the obtained text. Although the value of the contribution lies in the use of NLP techniques from the Amazon Comprehend tool, the paper lacks a formal results section, therefore it is difficult to determine the accuracy of the proposed tool. 

Although these studies constitute a relevant part of related work, e.g., by integrating useful tools such as Amazon Comprehend or detailing the use of techniques such as tokenization, the scientific literature does not address scenarios for the telecommunications field and even less in the context of a transparent digitalization of the RA. Therefore, and to the best of our knowledge, we can affirm that our work introduces a topic with a high degree of novelty.